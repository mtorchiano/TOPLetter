% \iffalse meta-comment
%
%  Copyright (C) 2015-2024 by Marco Torchiano <marco.torchiano@polito.it>
% -----------------------------------------------------------------------
% 
% This file may be distributed and/or modified under the conditions of
% the Apache License, either version 2.0 of this license
% or (at your option) any later version.  The latest version of this
% license is in:
% 
%    http://choosealicense.com/licenses/apache-2.0/
%
% \fi
%
% \iffalse
%<*driver>
\ProvidesFile{TOPletter.dtx}
%</driver>
%<class>\NeedsTeXFormat{LaTeX2e}
%<class>\ProvidesClass{TOPletter}
%<*class>
    [2024/09/12 v0.4.0 Corporate Letter PoliTo LaTeX class]
%</class>
%
%<*driver>
\documentclass{ltxdoc}
\usepackage{hyperref}
\usepackage{mdframed}
\EnableCrossrefs         
%\CodelineIndex
\PageIndex
\RecordChanges
\OnlyDescription
\begin{document}
  \DocInput{TOPletter.dtx}
\end{document}
%</driver>
% \fi
%
% \CheckSum{0}
% \CheckSum{292}
%
% \CharacterTable
%  {Upper-case    \A\B\C\D\E\F\G\H\I\J\K\L\M\N\O\P\Q\R\S\T\U\V\W\X\Y\Z
%   Lower-case    \a\b\c\d\e\f\g\h\i\j\k\l\m\n\o\p\q\r\s\t\u\v\w\x\y\z
%   Digits        \0\1\2\3\4\5\6\7\8\9
%   Exclamation   \!     Double quote  \"     Hash (number) \#
%   Dollar        \$     Percent       \%     Ampersand     \&
%   Acute accent  \'     Left paren    \(     Right paren   \)
%   Asterisk      \*     Plus          \+     Comma         \,
%   Minus         \-     Point         \.     Solidus       \/
%   Colon         \:     Semicolon     \;     Less than     \<
%   Equals        \=     Greater than  \>     Question mark \?
%   Commercial at \@     Left bracket  \[     Backslash     \\
%   Right bracket \]     Circumflex    \^     Underscore    \_
%   Grave accent  \`     Left brace    \{     Vertical bar  \|
%   Right brace   \}     Tilde         \~}
%
%
% \changes{v0.2}{2015/03/13}{Initial version}
% \changes{v0.2.1}{2015/03/13}{Minor changes and fixes to examples}
% \changes{v0.3.0}{2018/07/10}{Added possibility to add protocol number}
% \changes{v0.4.0}{2024/09/12}{Minor fixes + changed PolitTo logo}
%
% \GetFileInfo{TOPletter.dtx}
%
% \DoNotIndex{\newcommand,\newenvironment,\\, \bf,\begin, \cond, \condb}
% \DoNotIndex{\centering,\ClassError,\DeclareOption,\def,\definecolor}
% \DoNotIndex{\else,\end,\familydefault,\fancyfoot,\fancyhead,\fancyhf}
% \DoNotIndex{\fancypagestyle,\fi,\fontseries,\fontsize,\footrulewidth}
% \DoNotIndex{\footskip,\headheight,\headrulewidth,\href,\hspace}
% \DoNotIndex{\ifdefined,\IfFileExists,\IfLanguageName,\includegraphics}
% \DoNotIndex{\Large,\linebreak,\LoadClass,\makeatletter,\makeatother}
% \DoNotIndex{\makebox,\MessageBreak,\pagestyle,\parbox,\ProcessOption}
% \DoNotIndex{\protect,\raisebox,\relax,\renewcommand,\RequirePacakge}
% \DoNotIndex{\rule,\selectfont,\setlength,\sfdefault,\sffamily,\small}
% \DoNotIndex{\textwidth,\today,\topl@language,\toplData,\toplDest}
% \DoNotIndex{\toplEmail,\toplFax,\toplHome,\toplLuogo,\toplLogo}
% \DoNotIndex{\toplNome,\toplRuolo,\toplSignature,\toplStruttura}
% \DoNotIndex{\toplSubject,\toplSubjectLabel,\toplTelefono}
% \DoNotIndex{\toplTipoStruttura,\vspace,\color,\RequirePackage}
%
% \title{The \textsf{TOPletter} class\thanks{This document
%   corresponds to \textsf{TOPLetter}~\fileversion, dated \filedate.}}
% \author{Marco Torchiano \\ \texttt{marco.torchiano@polito.it}}
%
% \maketitle
%
% \section{Introduction}
%
% The \textsf{TOPletter} class provides the basic functions needed for 
% typesetting a letter using LaTeX and conforming to the official Corporate 
% Image guidelines for Politecnico di Torino~\footnote{
% \url{https://www.polito.it/en/polito/about-us/corporate-image}}. 
% The class can be used for letters written both in Italian and in English 
% languages.
%
%
%
% \section{Usage}
%
% The usage of the document class is extremely straightforward:
% \begin{itemize}
% \item the TOPletter document class must be declared as the base for
% the document. i.e. using |\documentclass{TOPletterA}|.
%
% \item the customization macros must be used to define the
%       details of the letter, i.e. organization details, personal
%		information, and letter subject.
%
% \item the body of the letter can be written as a regular \LaTeX document.
% \end{itemize}
% An example of a document using the \textsf{TOPletter} class is provided
% in section \ref{sec:example} below.
%
%
% \subsection{Organizational Unit}
% 
% \DescribeMacro{\dipartimentoDi}\noindent
% The name of the department can be specified using the 
% |\dipartimentoDi|\marg{department name} macro.
% 
% \noindent\DescribeMacro{\struttura}
% By default the organizational unit is a {\em department}, but
% it is possible to set the type of the organizational unit using
% |\struttura|\marg{structure type} macro.
% 
% \subsection{Personal Info}
% 
% \noindent\DescribeMacro{\nome}
% |\nome|\marg{name and surname}
% Defines the name of the person, it should include both first and
% last names as they will appear in the header of the letter and 
% at the end of the letter (after the date).
% 
% \noindent\DescribeMacro{\ruolo}
% |\ruolo|\marg{role}
% defines the role of the person, e.g. {\em Full Professor}.
% 
% \noindent\DescribeMacro{\telefono}
% |\telefono|\marg{phone number}
% defines the office phone number.
% 
% \noindent\DescribeMacro{\email}
% |\email|\marg{email}
% defines the email address of the person
% 
% The above information is mandatory and an error is reportedif any is missing.
%
% \noindent\DescribeMacro{\homepage}
% |\homepage|\marg{url}
% defines the (optional) URL of the personal home page.
% 
% \noindent\DescribeMacro{\fax}
% |\fax|\marg{fax}
% defines the (optional) fax number.
% 
% \noindent\DescribeMacro{\firma}
% |\firma|\marg{image file}
% defines an image file (typically a PNG) that contains a scanned signature of 
% the person.
%
% \subsection{Letter Details}
% 
% \noindent\DescribeMacro{\soggetto}
% |\soggetto|\marg{subject}
% defines the subject of the letter, that will appear, at the top of the letter
% preceded by the word "{\em Subject:}" (or "{\em Soggetto:}" if the language is 
% Italian).
%
% The above macro is the only mandatory for the letter details.
%
% \noindent\DescribeMacro{\destinatario}
% |\destinatario|\marg{recipient}
% defines the recipient of the letter, if no recipient is provided the 
% letter will be addressed to "{\em To whom it may concern}"
% (or "{\em A chi di competenza}" if the language is Italian).
% 
% \noindent\DescribeMacro{\luogo}
% |\luogo|\marg{place}
% defines the place that accompanies the date, it is by default "{\em Torino}" 
% 
% \noindent\DescribeMacro{\data}
% |\data|\marg{date}
% defines a date to be used at the bottom of the letter instead of the current date.
% 
% \noindent\DescribeMacro{\esoggetto}
% |\esoggetto|\marg{label}
% changes the word that precedes the subject topic, replaces the default "{\em Subject:}".
%
% \noindent\DescribeMacro{\protocollo}
% |\protocollo|\marg{label}
% define a protol number if empty, only the ``{\em Prot:}'' will appear at top left.
%
% \subsection{Logo}
% 
% \noindent\DescribeMacro{\logo}
% |\logo|\marg{image file}
% defines the image file containing the logo used in the header of the letter.
% The logo must be square with at least 400x400 pixel resolution.
% The default is the Politecnico di Torino logo.
%
% \subsection{Language}
% The document language can be specified as either |italian| or |english|
% as class parameter e.g. as |\documentclass[italian]{TOPletter}|.
% The default language is |italian|.
%
% The language can be detected using the |\IfLanguageName| macro to adapt the
% document contents to the specific language. This can be useful if a template
% file is used for both languages, for instance if we want to adapt the OU type
% and name according to the language we can add in the tex file the following:
%
% \begin{verbatim}
% \IfLanguageName{italian}{ %% department italian name:
%   \dipartimentoDi{Automatica e Informatica}
% }{%% department english name:
%   \dipartimentoDi{Control and Computer Engineering}
% }
% \end{verbatim}
% Automatically, based on the document language (defined as documentclass parameter),
% the correct naming for the department will be used.
%
%
%
% \section{Example}
% \label{sec:example}
% This section reports a very basic example of a short template used for writing a
% bilingual letter.
% \begin{mdframed}
% \begin{verbatim}
% \documentclass[english]{TOPletter}
% %% Load any additional package
% \usepackage{lipsum}
% 
% \IfLanguageName{italian}{ %% italian names
%   \dipartimentoDi{Disegno e Animazione}
%   \ruolo{Produttore}
% }{%% english names
%   \dipartimentoDi{Comics and Cartoons}
%   \ruolo{Producer}
% }
%
% %% personal details
% \nome{Walt DISNEY}
% \telefono{+1 555 123 4567}
% \email{walt.disney@polito.it}
% \homepage{http://www.polito.it/wdisney/}
%
% %% Subject
% \soggetto{Sample letter}
%
% \begin{document}
% %% Letter content
% This is the typical appearance of a PoliTo letter.
%
% \lipsum
% \end{document}
% \end{verbatim}
% \end{mdframed}
%
%
% \StopEventually{\PrintChanges\PrintIndex}
%
% \iffalse
%<*class>
% \fi
% \section{Implementation}
%
% This class is based on the standard class article, it uses an A4 paper
% format since it is the standard in EU, the default font size is set to 11pt.
%    \begin{macrocode}
\LoadClass[a4paper,11pt]{article}
%    \end{macrocode}
% The class is designe to work with two languages |italian| and |english|.
%
%    \begin{macrocode}
\def\topl@language{main=italian,english}
\DeclareOption{italian}{\def\topl@language{main=italian,english}}
\DeclareOption{english}{\def\topl@language{main=english,italian}}
\ProcessOptions\relax
%    \end{macrocode}
%
% \subsection{Used packages}
%    \begin{macrocode}
\RequirePackage[a4paper,bottom=8cm,top=2cm,left=2cm,right=2cm]{geometry}
\RequirePackage{graphicx}
\RequirePackage{color}
\RequirePackage{helvet}
\RequirePackage{changepage}
\RequirePackage{setspace}
\RequirePackage{fancyhdr}
\RequirePackage[utf8]{inputenc}
\RequirePackage{hyperref}
\RequirePackage[\topl@language]{babel}
\RequirePackage{iflang}
%    \end{macrocode}
%
% \subsection{Colors}
% Offical PoliTo colors defined in the Corporate Image Guidelines.
% 
%    \begin{macrocode}
\definecolor{orange}{cmyk}{0,.70,1.00,0} %C0/M70/Y100/K0
\definecolor{blue}{cmyk}{1.00,.80,0,.30} %C100/M80/Y0/K30
%    \end{macrocode}
% and we define accordingly the font faces and colors and a few
% macros to set text color quickly.
%
%    \begin{macrocode}
\renewcommand*\familydefault{\sfdefault}
\newcommand{\cond}[1]{\fontseries{mc}\selectfont {#1}}
\newcommand{\condb}[1]{\fontseries{bc}\selectfont {#1}}
\newcommand{\orange}[1]{\color{orange}{#1}}
\newcommand{\blue}[1]{\color{blue}{#1}}
\newcommand{\red}[1]{\color{red}{#1}}
%    \end{macrocode}
% \subsection{Headers and footer}
% Define the height of the head and the extra space at the bottom of the page:
%    \begin{macrocode}
\setlength{\headheight}{4.1cm}
\setlength{\footskip}{2cm}
%    \end{macrocode}
% The header part of the logo is define using the |fancyhdr| package.
%    \begin{macrocode}
\fancypagestyle{plain}{
\fancyhf{} % empty plus
%    \end{macrocode}
% The header contains only the left part, and it is equal on both
% even and odd pages. In particular it contains the square logo
% followed by the name of the department.
%    \begin{macrocode}
\fancyhead[LE,LO]{
\makebox[2.15cm]{\includegraphics[width=2cm,height=2cm]{\toplLogo}}
{\color{blue}\rule{0.4pt}{2cm}}
\hspace{1pt}
\parbox[b]{5cm}{
\condb{\fontsize{10.5pt}{10pt}\selectfont 
{\blue POLITECNICO\linebreak
	   \raisebox{3pt}{DI TORINO}}}\\
{\fontsize{7.5pt}{6pt}\selectfont \cond {\blue{%
\raisebox{-14pt}{\toplTipoStruttura}\\ \toplStruttura{} }}}%
}\\
%    \end{macrocode}
% After that we have the name and role of the person writing the letter
%    \begin{macrocode}
{\sffamily \small 
\vspace{0.8cm}
\hspace{2.5cm}\color{blue}{\cond{\toplRuolo{}}}\\
\hspace{2.5cm}\color{orange}{\condb{\toplNome{}}}}}
%    \end{macrocode}
% The footer is the same on odd and even pages and contains a reference to
% the structure and the address
%    \begin{macrocode}
\fancyfoot[LE,LO]{\small \cond{\color{blue}{{\bf \toplTipoStruttura \toplStruttura}\\
Politecnico di Torino - Corso Duca degli Abruzzi 24 - 10129 Torino - Italia\\
tel:\toplTelefono{} \ifdefined\toplFax\hspace{4pt} fax:\toplFax\fi} \\
\color{orange}{\href{mailto:\toplEmail}{\toplEmail}\hspace{4pt}%
 \ifdefined\toplHome\href{\toplHome}{\toplHome}%
 \fi \hspace{4pt} \href{http://www.polito.it}{http://www.polito.it}}}}
\renewcommand{\headrulewidth}{0pt}
\renewcommand{\footrulewidth}{0pt}
}
\pagestyle{plain}
%    \end{macrocode}
% \subsection{Language specific issues}
% The name of the structure is different in english and in italian,
% the two default version -- i.e. Dipartimento and Department -- are 
% predefined depending on the language:
%    \begin{macrocode}
\IfLanguageName{italian}{
	\newcommand{\toplTipoStruttura}{Dipartimento di }
	\newcommand{\toplDest}{A chi di competenza}
	\newcommand{\toplSubjectLabel}{Oggetto}
}{
	\newcommand{\toplTipoStruttura}{Department of }
	\newcommand{\toplDest}{To whom it may concern}
	\newcommand{\toplSubjectLabel}{Subject}
}
%    \end{macrocode}
% \subsection{Letter customization parameters}
% There are several parameters that can be used to customize the letter:
%    \begin{macrocode}
\newcommand{\dipartimentoDi}[1]{\def\toplStruttura{#1}}
\newcommand{\struttura}[1]{\def\toplTipoStruttura{#1}}
\newcommand{\ruolo}[1]{\makeatletter\def\toplRuolo{#1}\makeatother}
\newcommand{\nome}[1]{\makeatletter\def\toplNome{#1}\makeatother}
\newcommand{\telefono}[1]{\makeatletter\def\toplTelefono{#1}\makeatother}
\newcommand{\fax}[1]{\makeatletter\def\toplFax{#1}\makeatother}
\newcommand{\email}[1]{\makeatletter\def\toplEmail{#1}\makeatother}
\newcommand{\homepage}[1]{\def\toplHome{#1}}
\newcommand{\soggetto}[1]{\def\toplSubject{#1}}
\newcommand{\esoggetto}[1]{\def\toplSubjectLabel{#1}}
\newcommand{\destinatario}[1]{\def\toplDest{#1}}
\newcommand{\firma}[1]{\def\toplSignature{#1}}
\newcommand{\logo}[1]{\def\toplLogo{#1}}
\newcommand{\luogo}[1]{\def\toplLuogo{#1}}
\newcommand{\data}[1]{\def\toplData{#1}}
\newcommand{\protocollo}[1]{\def\toplProtocollo{#1}}
\def\toplLogo{LogoPolitoBlu.pdf}
\def\toplLuogo{Torino}
\def\toplData{\today}
%    \end{macrocode}
% \subsection{Checks}
% Several checks of completeness are performed at the beginning of the document.
%    \begin{macrocode}
\AtBeginDocument{
\ifdefined\toplEmail \else
\ClassError{TOPLetter} {Manca l'email}
{L'email deve essere definito con \protect\email{}.}
\fi
\ifdefined\toplTelefono \else
\ClassError{TOPLetter} {Manca il numero di telefono}
{Il numero di telefono deve essere definito con \protect\telefono{}.}
\fi
\ifdefined\toplNome \else
\ClassError{TOPLetter} {Manca il nome}
{Il nome della persona deve essere definito con \protect\nome{}.}
\fi
\ifdefined\toplRuolo \else
\ClassError{TOPLetter} {Manca il ruolo}
{Il ruolo deve essere definito con \protect\ruolo{}.}
\fi
\ifdefined\toplStruttura \else
\ClassError{TOPLetter} {Manca il nome della struttura}
{Il nome della struttura deve essere definito con \protect\struttura{}.}
\fi
\IfFileExists{\toplLogo} {} 
{
\ClassError{TOPLetter} {Il file di logo '\toplLogo' non e' stato trovato.}
{Il file deve avere come nome 'PoliLogoBlu.png' oppure essere definito
tramite il comando \protect\logo{}.\MessageBreak
Il file deve essere quadrato e con risoluzione sufficiente,
almeno 400x400 pixel.
}
}
%    \end{macrocode}
% \subsection{Begin of the letter}
% The initial part of the letter includes the recipient of  the letter
%    \begin{macrocode}
\ifdefined\toplProtocollo
Prot. \toplProtocollo
\vspace{1cm}
\fi
\begin{adjustwidth}{10cm}{}
	\toplDest
\end{adjustwidth}
\vspace{1cm}
%    \end{macrocode}
% After the recicpient the subject of the letter is specified
%    \begin{macrocode}
\ifdefined\toplSubject
\begin{center}
{\Large {\bf \toplSubjectLabel: \toplSubject }}
\end{center}
\fi
}
%    \end{macrocode}
% \subsection{End of the letter}
% at the end of the letter we have the place and date, followed by the signature,
% if defined.
%    \begin{macrocode}
\AtEndDocument{
\vspace{2cm}
\hspace{.6\textwidth}
\parbox{5cm}{
\centering
\toplLuogo, \toplData\\
\toplNome\\
\ifdefined\toplSignature
	\includegraphics[width=4cm]{\toplSignature}
\fi
}
}
%    \end{macrocode}
%% \iffalse
%</class>
% \fi
% \Finale
% \iffalse
%<*sampleIt>
% \fi
\documentclass{TOPletter}
%% Carica package necessari
\usepackage{lipsum}

%% dipartimento
\dipartimentoDi{Disegno e Animazione}

%% dettagli personali
\ruolo{Fumettista }
\nome{Walt DISNEY}
\telefono{+1 555 123 4567}
\email{walt.disney@polito.it}
\homepage{http://www.polito.it/wdisney/}
\firma{Walt_Disney_1942_signature.pdf}
%% Subject
\soggetto{Esempio di lettera}
\protocollo{}
\begin{document}
%% Letter content
Ecco l'aspetto di una lettera PoliTo...

\lipsum

\end{document}
% \iffalse
%</sampleIt>
%<*sampleEn>
% \fi
\documentclass[english]{TOPletter}
%% Load any additional package
\usepackage{lipsum}

\IfLanguageName{italian}{ %% italian names
  \dipartimentoDi{Disegno e Animazione}
  \ruolo{Produttore}
}{%% english names
  \dipartimentoDi{Comics and Cartoons}
  \ruolo{Producer}
}

%% personal details
\nome{Walt DISNEY}
\telefono{+1 555 123 4567}
\email{walt.disney@polito.it}
\homepage{http://www.polito.it/wdisney/}

%% Subject
\soggetto{Sample letter}

\begin{document}
%% Letter content
This is the typical appearance of a PoliTo letter.

\lipsum
\end{document}
% \iffalse
%</sampleEn>
% \fi
\endinput
